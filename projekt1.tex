\documentclass[12pt]{article}
\usepackage{amsmath}
\usepackage{amssymb}
\usepackage{amsfonts}
\usepackage[polish]{babel}
\usepackage[T1]{fontenc}
\usepackage[dvips]{graphicx}
\usepackage[cp1250]{inputenc}
\usepackage{caption}
\usepackage{enumerate}


\captionsetup[table]{name=Tabela}


\textheight 23.2 cm

\textwidth 6.0 in

\hoffset = -0.5 in

\voffset = -2.4 cm

\hyphenation{me-to-dy la-bo-ra-to-rium}

\begin{document}

%hę?
%\thispagestyle{empty}

\vspace*{3ex}
\begin{flushright}
{\large 12 grudnia 2019}
\end{flushright}

\begin{flushleft}
{\large Jakub Kujawa\\
Miko\l{}aj Kowalski\\
Grupa G8}
\end{flushleft}

\hskip3cm

\begin{center}

\Large {\bf Rozwi\k{a}zywanie uk\l{}adu r\'owna\'n liniowych XA=B, gdzie $A\in \mathbb R^\mathrm{n\times n}$, $B\in \mathbb R^\mathrm{n\times n}$, zmodyfikowan\k{a} metod\k{a} Doolittle'a (tj. poprzez rozk\l{}ad $A=UL$, gdzie U jest macierz\k{a} tr\'ojk\k{a}tn\k{a} g\'orn\k{a}, a L macierz\k{a} tr\'ojk\k{a}tn\k{a} doln\k{a} z jedynkami na g\l{}\'ownej przek\k{a}tnej). Wyznaczanie macierzy $A^{-1}$ oraz $det(A)$ na podstawie rozk\l{}adu.}\vskip2ex

{\large Projekt nr 1}

\end{center}

\vskip20ex



\section{Opis metody}

\noindent Rozwi\k{a}zanie uk\l{}adu r\'owna\'n XA=B, gdzie $A\in \mathbb R^\mathrm{n\times n}$, $B\in \mathbb R^\mathrm{n\times n}$, zmodyfikowan\k{a} metod\k{a} Doolitle'a opiera si\k{e} na wyznaczeniu macierzy $L\in \mathbb R^\mathrm{n\times n}$ oraz $U\in \mathbb R^\mathrm{n\times n}$ takich, \.ze L jest macierz\k{a} tr\'ojk\k{a}tn\k{a} doln\k{a} z jedynkami na gl\'ownej przek\k{a}tnej, a U macierz\k{a} tr\'ojk\k{a}tn\k{a} g\'orn\k{a} oraz A=UL.


\noindent Niech:
\[U=\begin{pmatrix}
u_{11} & u_{12} & u_{13} & \ldots & u_{1n} \\
0 & u_{22} & u_{23} & \ldots & u_{2n} \\
0 & 0 & u_{33} & \ldots & u_{3n} \\
\vdots & \vdots & \vdots & \ddots & \vdots \\
0 & 0 & 0 & \ldots & u_{nn}
\end{pmatrix}.
\]
oraz 
\[L=\begin{pmatrix}
1 & 0 & 0 & \ldots & 0 \\
l_{21} & 1 & 0 & \ldots & 0 \\
l_{31} & l_{32} & 1 & \ldots & 0 \\
\vdots & \vdots & \vdots & \ddots & \vdots \\
l_{n1} & l_{n2} & l_{n3} & \ldots & 1
\end{pmatrix}.
\]
Poszczeg\'olne elementy macierzy wyznaczamy ze wzor\'ow:
\begin{equation}
u_{ij}=a_{ij}-\sum_{k=j+1}^n u_{ik}l_{kj} \label{U}
\end{equation}

oraz
 
\begin{equation}
l_{ij}=\dfrac{a_{ij}-\sum_{k=i+1}^n u_{ik}l_{kj}}{u_{ii}} \label{L}
\end{equation}
\\
gdzie wz\'or (\ref{U}) stosujemy dla $i \leq j$, a wz\'or (\ref{L}) dla $i > j$.
Schemat wyznaczania opiera si\k{e} na naprzemiennym obliczaniu warto\'sci element\'ow macierzy U w n-tej kolumnie, warto\'sci element\'ow w n-tym wierszu macierzy L, n-1-szej kolumnie macierzy U, n-1-szym wierszu macierzy L itd.
\\
Po wyznaczeniu macierzy L oraz U nale\.zy rozwi\k{a}za\'c nast\k{e}puj\k{a}ce r\'ownania:
\begin{equation}
YL=B \label{Y}
\end{equation}
oraz
\begin{equation}
XU=Y \label{X}
\end{equation}
\\
gdzie $Y\in \mathbb R^\mathrm{n\times n}$. 
\\
Macierz Y z r\'ownania (\ref{Y}) wyznacza si\k{e} za pomoc\k{a} wzoru
\[
y_{ij}=b_{ij}-\sum_{k=j+1}^n y_{ik}l_{kj} 
\]
wyznaczaj\k{a}c kolejne kolumny macierzy zaczynaj\k{a}c od n-tej. 
\\
 Natomiast macierz X z r\'ownania (\ref{X}) wyznacza si\k{e} za pomoc\k{a} wzoru
 \[
 x_{ij}=\frac{y_{ij}-\sum_{k=1}^{i-1} x_{ik}u_{ki}}{u_{jj}}
 \]
 wyznaczaj\k{a}c kolejne kolumny macierzy zaczynaj\k{a}c od pierwszej. 
\\
\\
Wyznacznik macierzy A wyznaczymy dzi\k{e}ki zale\.zno\'sci:
\[
det(A)=det(UL)=det(U)*det(L)
\]
Natomiast z w\l{}asno\'sci macierzy tr\'ojk\k{a}tnych mamy:
\[
det(L)=\prod_{i=1}^n l_{i,i}=1
\]
Zatem 
\[
det(A)=det(U)
\]
Wyznaczaj\k{a}c odwrotno\'s\'c macierzy A r\'ownie\.z skorzystamy z wyznaczonych wcze\'sniej macierzy L i U:
\[
A^{-1}=(UL)^{-1}=L^{-1}U^{-1}
\]
gdzie $L^{-1}$ i $U^{-1}$ istniej\k{a} tylko, gdy $det(U)\neq 0$, czyli gdy \.zadna warto\'s\'c na gl\'ownej przek\k{a}tnej macierzy U nie jest r\'owny 0.
\\
Niech 
\[
L^{-1}=\begin{pmatrix}
l_{11}' & 0 & 0 & \ldots & 0 \\
l_{21}' & l_{22}' & 0 & \ldots & 0 \\
l_{31}' & l_{32}' & l_{33}' & \ldots & 0 \\
\vdots & \vdots & \vdots & \ddots & \vdots \\
l_{n1}' & l_{n2}' & l_{n3}' & \ldots & l_{nn}'
\end{pmatrix}.
\]
oraz
\[
U^{-1}=\begin{pmatrix}
u_{11}' & u_{12}' & u_{13}' & \ldots & u_{1n}' \\
0 & u_{22}' & u_{23}' & \ldots & u_{2n}' \\
0 & 0 & u_{33}' & \ldots & u_{3n}' \\
\vdots & \vdots & \vdots & \ddots & \vdots \\
0 & 0 & 0 & \ldots & u_{nn}'
\end{pmatrix}.
\]
Z w\l{}asno\'sci macierzy tr\'ojk\k{a}tnych mamy:
\[
l_{ii}'=1
\]
oraz
\[
u_{ii}'=\frac{1}{u_{ii}}
\]
Pozosta\l{}e elementy macierzy $L^{-1}$ oraz $U^{-1}$ wyznaczymy, wykorzystuj\k{a}c wzory:
\[
l_{ij}'=-\sum_{k=j+1}^n l_{ik}l_{kj}'
\]
oraz
\[
u_{ij}'=-\frac{\sum_{k=i+1}^n u_{ik}u_{kj}'}{u_{ii}}
\]
\\
gdzie elementy macierzy $L^{-1}$ wyznaczamy wierszami od pierwszego do n-tego, a elementy macierzy $U^{-1}$ wierszami od n-tego do pierwszego. 
\\
\\
Wykorzystuj\k{a}c program do wyliczenia macierzy X nie unikniemy b\l{}\k{e}d\'ow obliczeniowych. Ich analiz\k{e} oprzemy na wyliczeniach:
\begin{enumerate}[i)]
\item wska\'znika uwarunkowania macierzy A:
\[
cond(A)=\Vert A^{-1}\Vert  \Vert A \Vert
\]
\item b\l{}\k{e}du rozk\l{}adu:
\[
e_{dec}=\frac{\Vert A-UL \Vert}{\Vert A \Vert}
\]
\item b\l{}\k{e}du wzgl\k{e}dnego:
\[
e_{rel}=\frac{\Vert X-Z \Vert}{\Vert Z \Vert}
\]
\item wsp\'o\l{}czynnika stabilo\'sci:
\[
wsp_{stab}=\frac{\Vert X-Z \Vert}{\Vert Z \Vert cond(A)}
\]
\item wsp\'o\l{}czynnika poprawno\'sci:
\[
wsp_{popr}=\frac{\Vert B-XA \Vert}{\Vert A \Vert \Vert X \Vert}
\]
\item jakiegos kurna wpolczynnika prawego:
\[
r_{R}=\frac{\Vert AA^{-1}-I\Vert}{\Vert A \Vert \Vert A^{-1} \Vert}
\]
\item jakiegos kurna wspolczynnika lewego:
\[
r_{L}=\frac{\Vert A^{-1}A-I\Vert}{\Vert A \Vert \Vert A^{-1} \Vert}
\]
gdzie Z jest dok\l{}adnym rozwi\k{a}zaniem uk\l{}adu, X rozwi\k{a}zaniem obliczonym numerycznie przybli\.zeniem wyznaczonym naszym algorytmem,a $A^{-1}$ wyznaczon\k{a} przez nas odwrotno\'sci\k{a} macierzy A. 


\end{enumerate}

\vskip20pt

\section{Opis programu obliczeniowego}

W czasie tworzenia programu obliczeniowego zosta\l{}y stworzone nast\k{e}puj\k{a}ce funkcje:
\begin{enumerate}
\item mdoolittle(A)=[U,L]:
\\
przyjmuje za argument macierz kwadratow\k{a} A i wyznacza jej rozk\l{}ad UL

\item invmd(A)=[Ai]:
\\
przyjmuje za argument macierz kwadratow\k{a} A i wyznacza jej odwrotno\'s\'c za pomoc\k{a} rozk\l{a}du UL
 
\item solvemd(A,B)=[X]:
\\
przyjmuje za argument macierze kwadratowe o tych samych wymiarach A oraz B i wyznacza rozwi\k{a}zanie uk\l{}adu nier\'owno\'sci XA=B za pomoc\k{a} rozk\l{}adu UL

\item detmd(A)=d:
\\
przyjmuje za argument macierz kwadratow\k{a} A i wyznacza jej wyznacznik za pomoc\k{a} rozk\l{}adu UL

\item condmd(A,p)=c:
\\
przyjmuje za argument macierz kwadratow\k{a} A oraz p-norm\k{e} i wyznacza wska\'znik uwarunkowania macierzy A korzystaj\k{a}c z p-normy i funkcji invmd do wyznaczenia $A^{-1}$

\item edec(A,p)=e:
\\
przyjmuje za argument macierz kwadratow\k{a} A oraz p-norm\k{e} i wyznacza b\l{}\k{a}d rozk\l{}adu macierzy A zmodyfikowan\k{a} metod\k{a} Doolittle'a korzystaj\k{a}c z p-normy

\item erel(A,B,p)=e:
\\
przyjmuje za argument macierz kwadratow\k{a} A, macierz kwadratow\k{a} B oraz p-norm\k{e} i wyznacza b\l{a}d wzgl\k{e}dny (korzystaj\k{a}c z p-normy) rozwi\k{a}zania uk\l{}adu XA=B zmodyfikowan\k{a} metod\k{a} Doolittle'a

\item wsppopr(A,B,p)=w:
\\
przyjmuje za argument macierz kwadratow\k{a} A, macierz kwadratow\k{a} B oraz p-norm\k{e} i wyznacza wsp\'o\l{}czynnik poprawno\'sci (korzystaj\k{a}c z p-normy) rozwi\k{a}zania uk\l{}adu XA=B zmodyfikowan\k{a} metod\k{a} Doolitle'a

\item wspstab(A,B,p)=w:
\\
przyjmuje za argument macierz kwadratow\k{a} A, macierz kwadratow\k{a} B oraz p-norm\k{e} i wyznacza wsp\'o\l{}czynnik stabilno\'sci (korzystaj\k{a}c z p-normy) rozwi\k{a}zania uk\l{}adu XA=B zmodyfikowan\k{a} metod\k{a} Doolitle'a wzgl\k{e}dem rozwi\k{a}zania $X=B\cdot A^{-1}$

\item rR
\item rL 

\end{enumerate}

\vskip20pt

\section{Przyk\l ady obliczeniowe}

\noindent 

/ToDo 


\section{Analiza wynik\'ow}

Cos tu sie zanalizuje

{\small
\begin{thebibliography}{11}

\bibitem{dahlquist: 83} {\rm G.~Dahlquist and \AA.~Bj\"orck},
Metody numeryczne, PWN, Warszawa, 1983.

\bibitem{jankowscy: 81} {\rm J. i M. Jankowscy}, Przegl\k ad metod i algorytm\'ow
numerycznych, cz.\ 1, WNT, Warszawa, 1981.

\end{thebibliography}
}



\vskip3ex
\begin{center}
\hspace*{-2ex}
\begin{tabular}{|l|} \hline
\\
Wszelkie pytania i wnioski prosimy kierowa\'c na
adres:\\
\\
wrubelki@wp.pl\\
\\
\hline
\end{tabular}

\end{center}
\end{document}