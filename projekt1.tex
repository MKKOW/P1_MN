\documentclass[12pt]{article}
\usepackage{amsmath}
\usepackage{amssymb}
\usepackage{amsfonts}
\usepackage[polish]{babel}
\usepackage[T1]{fontenc}
\usepackage[dvips]{graphicx}
\usepackage[cp1250]{inputenc}
\usepackage{caption}
\usepackage{enumerate}


\captionsetup[table]{name=Tabela}


\textheight 23.2 cm

\textwidth 6.0 in

\hoffset = -0.5 in

\voffset = -2.4 cm

\hyphenation{me-to-dy la-bo-ra-to-rium}

\begin{document}

%hę?
%\thispagestyle{empty}

\vspace*{3ex}
\begin{flushright}
{\large 12 grudnia 2019}
\end{flushright}

\begin{flushleft}
{\large Jakub Kujawa\\
Miko\l{}aj Kowalski\\
Grupa G8}
\end{flushleft}

\hskip3cm

\begin{center}

\Large {\bf Rozwi\k{a}zywanie uk\l{}adu r\'owna\'n liniowych XA=B, gdzie $A\in \mathbb R^\mathrm{n\times n}$, $B\in \mathbb R^\mathrm{n\times n}$, zmodyfikowan\k{a} metod\k{a} Doolittle'a (tj. poprzez rozk\l{}ad $A=UL$, gdzie U jest macierz\k{a} tr\'ojk\k{a}tn\k{a} g\'orn\k{a}, a L macierz\k{a} tr\'ojk\k{a}tn\k{a} doln\k{a} z jedynkami na g\l{}\'ownej przek\k{a}tnej). Wyznaczanie macierzy $A^{-1}$ oraz $det(A)$ na podstawie rozk\l{}adu.}\vskip2ex

{\large Projekt nr 1}

\end{center}

\vskip20ex



\section{Opis metody}

\noindent Rozwi\k{a}zanie uk\l{}adu r\'owna\'n XA=B, gdzie $A\in \mathbb R^\mathrm{n\times n}$, $B\in \mathbb R^\mathrm{n\times n}$, zmodyfikowan\k{a} metod\k{a} Doolitle'a opiera si\k{e} na wyznaczeniu macierzy $L\in \mathbb R^\mathrm{n\times n}$ oraz $U\in \mathbb R^\mathrm{n\times n}$ takich, \.ze L jest macierz\k{a} tr\'ojk\k{a}tn\k{a} doln\k{a} z jedynkami na gl\'ownej przek\k{a}tnej, a U macierz\k{a} tr\'ojk\k{a}tn\k{a} g\'orn\k{a} oraz A=UL.


\noindent Niech:
\[U=\begin{pmatrix}
u_{1,1} & u_{1,2} & u_{1,3} & \ldots & u_{1,n} \\
0 & u_{2,2} & u_{2,3} & \ldots & u_{2,n} \\
0 & 0 & u_{3,3} & \ldots & u_{3,n} \\
\vdots & \vdots & \vdots & \ddots & \vdots \\
0 & 0 & 0 & \ldots & u_{n,n}
\end{pmatrix}.
\]
oraz 
\[L=\begin{pmatrix}
1 & 0 & 0 & \ldots & 0 \\
l_{2,1} & 1 & 0 & \ldots & 0 \\
l_{3,1} & l_{3,2} & 1 & \ldots & 0 \\
\vdots & \vdots & \vdots & \ddots & \vdots \\
l_{n,1} & l_{n,2} & l_{n,3} & \ldots & 1
\end{pmatrix}.
\]
Poszczeg\'olne elementy macierzy wyznaczamy ze wzor\'ow:
\begin{equation}
u_{i,j}=a_{i,j}-\sum_{k=j+1}^n u_{i,k}*l_{k,j} \label{U}
\end{equation}

oraz
 
\begin{equation}
l_{i,j}=\dfrac{a_{i,j}-\sum_{k=i+1}^n u_{i,k}*l_{k,j}}{u_{i,i}} \label{L}
\end{equation}
\\
gdzie wz\'or (\ref{U}) stosujemy dla $i \leq j$, a wz\'or (\ref{L}) dla $i > j$.
Schemat wyznaczania opiera si\k{e} na naprzemiennym obliczaniu warto\'sci element\'ow macierzy U w n-tej kolumnie, warto\'sci element\'ow w n-tym wierszu macierzy L, n-1-szej kolumnie macierzy U, n-1-szym wierszu macierzy L itd.
\\
Po wyznaczeniu macierzy L oraz U nale\.zy rozwi\k{a}za\'c nast\k{e}puj\k{a}ce r\'ownania:
\begin{equation}
YL=B \label{backward}
\end{equation}
oraz
\begin{equation}
XU=Y \label{forward}
\end{equation}
\\
gdzie $Y\in \mathbb R^\mathrm{n\times n}$. 
\\
Macierz Y z r\'ownania (\ref{backward}) wyznacza si\k{e} za pomoc\k{a} algorytmu Backward Substitution. Natomiast macierz X z r\'ownania (\ref{forward}) wyznacza si\k{e} za pomoc\k{a} algrytmu Forward Substitution.
\\
\\
Wyznacznik macierzy A wyznaczymy dzi\k{e}ki prostej zale\.zno\'sci:
\[
det(A)=det(UL)=det(U)*det(L)
\]
Natomiast z w\l{}asno\'sci macierzy tr\'ojk\k{a}tnych mamy:
\[
det(U)=\prod_{i=1}^n u_{i,i}
\]
oraz
\[
det(L)=1
\]
Zatem 
\[
det(A)=\prod_{i=1}^n u_{i,i}
\]
Wyznaczaj\k{a}c odwrotno\'s\'c macierzy A r\'ownie\.z skorzystamy z wyznaczonych wcze\'sniej macierzy L i U:
\[
A^{-1}=(UL)^{-1}=L^{-1}U^{-1}
\]
\\
Natomiast $L^{-1}$ oraz $U^{-1}$ wyznaczymy, wykorzystuj\k{a}c kolejny raz algorytmy odpowiednio Backward oraz Forward Substitution. Oczywi\'scie odwrotno\'sci macierzy wyznaczymy pod warunkiem, \.ze ich wyznacznik jest r\'o\.zny od zera. 
\\
\\
Wykorzystuj\k{a}c program do wyliczenia macierzy X nie unikniemy b\l{}\k{e}d\'ow obliczeniowych. Ich analiz\k{e} oprzemy na wyliczeniach:
\begin{enumerate}[i)]
\item wska\'znika uwarunkowania macierzy A:
\[
cond(A)=\Vert A^{-1}\Vert  \Vert A \Vert
\]
\item b\l{}\k{e}du rozk\l{}adu:
\[
e_{dec}=\frac{\Vert A-BC \Vert}{\Vert A \Vert}
\]
\item b\l{}\k{e}du wzgl\k{e}dnego:
\[
e_{rel}=\frac{\Vert X-Z \Vert}{\Vert Z \Vert}
\]
\item wsp\'o\l{}czynnika stabilo\'sci:
\[
wsp_{stab}=\frac{\Vert X-Z \Vert}{\Vert Z \Vert cond(A)}
\]
\item wsp\'o\l{}czynnika poprawno\'sci:
\[
wsp_{popr}=\frac{\Vert B-AX \Vert}{\Vert A \Vert \Vert X \Vert}
\]
\item jakiegos kurna wpolczynnika prawego:
\[
r_{R}=\frac{\Vert AA^{-1}-I\Vert}{\Vert A \Vert \Vert A^{-1} \Vert}
\]
\item jakiegos kurna wspolczynnika lewego:
\[
r_{L}=\frac{\Vert A^{-1}A-I\Vert}{\Vert A \Vert \Vert A^{-1} \Vert}
\]
gdzie Z jest dok\l{}adnym rozwi\k{a}zaniem uk\l{}adu, X rozwi\k{a}zaniem obliczonym numerycznie przybli\.zeniem wyznaczonym naszym algorytmem,a $A^{-1}$ wyznaczon\k{a} przez nas odwrotno\'sci\k{a} macierzy A. 


\end{enumerate}

\vskip20pt

\section{Opis programu obliczeniowego}


/TODO
\\
jak wyskakuje w programie

\vskip20pt

\section{Przyk\l ady obliczeniowe}

\noindent 

Najwa\.zniejsze wyniki zestawiono w tabeli \ref{Tabela z wynikami algorytmu 1}.

\bigskip

\begin{table}[h!]
\caption{\footnotesize Moje wyniki} %\vskip1ex
\renewcommand{\arraystretch}{1.1}
\centering\begin{tabular}{|c|c|c|c|c|}
\hline $N$ & rozwi\k azanie & rozwi\k azanie & b\l \k ad \\
& dok\l adne & przybli\.zone & \\
\hline 10 & 1.6396 & 1.6395 & $1.6597e\!-\!04$ \\
\hline 12 & 6.4839 & 6.4839 & $6.2312e\!-\!05$ \\
\hline 14 & 6.4597 & 6.4597 & $8.8147e\!-\!05$ \\
\hline 16 & 1.0575 & 1.0674 & $1.2730e\!-\!04$ \\
\hline 18 & 3.0060 & 3.0061 & $4.3985e\!-\!04$ \\
\hline 20 & 7.1352 & 7.1351 & $7.6212e\!-\!04$ \\
\hline
\end{tabular}
\label{Tabela z wynikami algorytmu 1}
\end{table}


\vskip2ex

Jako\'s ma\l o tych przyk\l ad\'ow. O wiem, dodam obrazek.


\begin{center}
\includegraphics[width=10cm]{obrazek.eps}
\nopagebreak

{\footnotesize Rysunek 1: Przedstawia wykres oszacowania b\l \k edu\\
dla pewnego algorytmu.}
\end{center}

Teraz dodam go jeszcze raz, ale inaczej niż poprzednio.

\begin{figure}[!h]

\centering\includegraphics[width=10cm]{obrazek.eps}

\vskip-1.5ex

\caption{\footnotesize Przedstawia to samo, co poprzedni obrazek.}
\label{obrazek}
\end{figure}

Tym razem obrazek automatycznie dostał numer, do którego można się
łatwo odwoływać przy użyciu etykiety i instrukcji
,,$\backslash$ref''. Numer nie zgadza się z poprzednim, gdyż LaTeX
sam numeruje tylko te obrazki, które zostały utworzone za pomocą
środowiska ,,figure'' i dostały numer (przez użycie funkcji
,,$\backslash$caption''). Dodatkową zaletą drugiego sposobu
wstawiania rysunków jest fakt, że treść w obrębie środowiska
,,figure'' lub ,,table'' jest traktowana jako całość i nie
zostanie podzielona tak, jak to się zdarzyło z pierwszym
obrazkiem. Mogą natomiast pojawić się puste przestrzenie. Dotyczy
to również tabel.


\section{Analiza wynik\'ow}

Metoda zdaje si\k e dzia\l a\'c w wi\k ekszo\'sci przypadk\'ow.
Niekiedy jednak nie dzia\l a, lub dzia\l a niepoprawnie. Przyczyny
tego zjawiska są mi bliżej nieznane. Mo\.ze te elektrolityczne
krasnoludki, kt\'ore tak naprawd\k e wykonuj\k a obliczenia w
komputerze, czasami strajkuj\k a. Trzeba je lepiej karmi\'c.


{\small
\begin{thebibliography}{11}

\bibitem{dahlquist: 83} {\rm G.~Dahlquist and \AA.~Bj\"orck},
Metody numeryczne, PWN, Warszawa, 1983.

\bibitem{jankowscy: 81} {\rm J. i M. Jankowscy}, Przegl\k ad metod i algorytm\'ow
numerycznych, cz.\ 1, WNT, Warszawa, 1981.

\end{thebibliography}
}



\vskip3ex
\begin{center}
\hspace*{-2ex}
\begin{tabular}{|l|} \hline
\\
Wszelkie pytania i wnioski prosimy kierowa\'c na
adres:\\
\\
wrubelki@wp.pl\\
\\
\hline
\end{tabular}

\end{center}
\end{document}